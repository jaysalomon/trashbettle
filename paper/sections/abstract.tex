\begin{abstract}
This paper presents a theoretical framework and computational validation for a new class of potentially autonomous bio-hybrid organisms designed to achieve energy independence through environmental waste consumption. We propose a multi-functional flow lattice architecture that could address fundamental challenges in mobile robotics related to energy efficiency, system complexity, and mass. The key theoretical insight is a shift in thermodynamic strategy: from heat retention, which is common in industrial systems, to optimized heat delivery for integrated biological processes. This approach, using distributed micro-combustion, is predicted through simulations to achieve over 95\% thermal coupling efficiency. By proposing a functional, survival-driven control system with this novel internal architecture, we establish a theoretical pathway toward potentially creating autonomous artificial life forms. Computational simulations across four development cycles predict a 3× improvement in heat delivery efficiency for 4mm versus 12mm combustion chambers, supporting the theoretical scaling relationship $\eta_{delivery} \propto 1/d$. Monte Carlo analysis projects a median daily energy surplus of 0.62 kWh with 99.36\% reliability. Proposed experimental validation protocols are presented for future verification of these computational predictions.
\end{abstract}